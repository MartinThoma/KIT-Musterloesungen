\section*{Aufgabe 6}
Für den bivariaten Zufallsvektor $X$ gelte
\[X = \begin{pmatrix}X_1\\X_2\end{pmatrix} \sim N_2(\mu, \Sigma) \text{ mit } \mu=\begin{pmatrix}1\\1\end{pmatrix}, \Sigma = \begin{pmatrix}1 & c\\c & 1\end{pmatrix}\]

\begin{enumerate}[label=(\alph*)]
    \item Welche Werte kann $c$ annehmen? Welche Werte kann der Korrelationskoeffizient $\rho(X_1, X_2)$ annehmen? Begründen Sie ihre Antwort.
    \item Bestimmen Sie die Verteilung von $V = X_1 - X_2$.
    \item Es gelte $-1 \leq c \leq 1$. Es sei $U = X_1 + k X_2$ mit $k \in \mathbb{R}$. Für welche
          Werte von $k$ sind $U$ und $V$ stochastisch unabhängig?
\end{enumerate}
