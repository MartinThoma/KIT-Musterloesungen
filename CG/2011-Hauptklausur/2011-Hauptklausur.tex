\documentclass[a4paper]{scrartcl}
\usepackage[ngerman]{babel}
\usepackage[utf8]{inputenc}
\usepackage{amssymb,amsmath}
\usepackage{graphicx}
\usepackage[inline]{enumitem}
\setlist{noitemsep}
\usepackage[binary-units=true]{siunitx}
\usepackage{hyperref}
\usepackage{parskip}
\usepackage[nameinlink,noabbrev,ngerman]{cleveref} % has to be after hyperref
\usepackage[colorinlistoftodos]{todonotes}
\usepackage{nicefrac}  % for \nicefrac{1}{3}
\usepackage{csquotes}  % for \enquote{what you want to quote}
\usepackage{booktabs}  % for \toprule, \midrule and \bottomrule
\usepackage{minted} % needed for the inclusion of source code

% for \begin{enumerate}[label=(\Alph*)], see http://tex.stackexchange.com/a/129960/5645
\usepackage{enumitem}

\setcounter{secnumdepth}{2}
\setcounter{tocdepth}{2}

\usepackage{wasysym}  % For \CheckedBox
\usepackage{microtype}

\begin{document}
\selectlanguage{ngerman}
\title{2011 Hauptklausur (WS 2010/11)}


\setcounter{section}{1}
%%%%%%%%%%%%%%%%%%%%%%%%%%%%%%%%%%%%%%%%%%%%%%%%%%%%%%%%%%%%%%%%%%%%%%%%%%%%%%
\section*{Aufgabe 1: Wahrnehmung und Farbräume}
\subsection*{Teilaufgabe 1a}
\textit{Welche Eigenschaft der menschlichen Wahrnehmung wird durch das Weber-Fechner-Gesetz beschrieben?}

Das Weber-Fechner-Gesetz macht eine Aussage über die subjektiv empfundene
Stärke von Sinneseindrücken im Abhängigkeit von der Intensität des
Helligkeitsunterschiedes.

Es wird die Eigenschaft, dass die Stärke des Sinneseindrucks von der Intensität
logarithmisch abhängt beschrieben.


\subsection*{Teilaufgabe 1b}
\textit{Was ist der Gamut eines Monitors?}

Der Gamut eines Monitors entspricht dem Spektrum der darauf darstellbaren
Farben.

\subsection*{Teilaufgabe 1c}
\begin{tabular}{p{8cm}llll}\toprule
Aussage                                                  & RGB & CMY & HSV & CIE xyY \\\midrule
Der Farbraum ist additiv.                                & \CheckedBox & $\square$   & $\square$   & $\square$ \\
Der Farbraum ist subtraktiv.                             & $\square$   & \CheckedBox & $\square$   & $\square$ \\
Der Farbraum ist multiplikativ.                          & $\square$   & \CheckedBox & $\square$   & $\square$ \\
Der Farbraum trennt Luminanz von Chrominanz.             & $\square$   & $\square$   & \CheckedBox & $\square$ \\
Der Farbraum kann alle sichtbaren Farben repräsentieren. & $\square$   & $\square$   & ~           & \CheckedBox       \\
Der Farbraum wird nativ auf Peripheriegeräten verwendet. & \CheckedBox & \CheckedBox & $\square$   & $\square$ \\\bottomrule
\end{tabular}

\section*{Aufgabe 2}
TODO

\section*{Aufgabe 3}
TODO

\section*{Aufgabe 4}
TODO

\section*{Aufgabe 5}
TODO

\section*{Aufgabe 6}
TODO

\section*{Aufgabe 7}
TODO

\section*{Aufgabe 8}
TODO

\section*{Aufgabe 9}
TODO

\section*{Aufgabe 10}
\subsection*{Teilaufgabe 10a}
TODO

\subsection*{Teilaufgabe 10b}
TODO

\subsection*{Teilaufgabe 10c}
\begin{figure}[h]
    \centering
    \includegraphics*[width=0.8\linewidth, keepaspectratio]{10c.png}
    \caption{Skizze zu Aufgabe 10c}
    \label{fig:10c}
\end{figure}

\end{document}
