\documentclass[a4paper]{scrartcl}
\usepackage[ngerman]{babel}
\usepackage[utf8]{inputenc}
\usepackage{amssymb,amsmath}
\usepackage{graphicx}
\usepackage[inline]{enumitem}
\setlist{noitemsep}
\usepackage[binary-units=true]{siunitx}
\usepackage{hyperref}
\usepackage{parskip}
\usepackage[nameinlink,noabbrev,ngerman]{cleveref} % has to be after hyperref
\usepackage{nicefrac}  % for \nicefrac{1}{3}
\usepackage{csquotes}  % for \enquote{what you want to quote}
\usepackage{booktabs}  % for \toprule, \midrule and \bottomrule
\usepackage{minted} % needed for the inclusion of source code

\usepackage{tikz}
\usetikzlibrary{intersections}

% for \begin{enumerate}[label=(\Alph*)], see http://tex.stackexchange.com/a/129960/5645
\usepackage{enumitem}

\setcounter{secnumdepth}{2}
\setcounter{tocdepth}{2}

\usepackage{wasysym}  % For \CheckedBox
\usepackage{microtype}

\begin{document}
\selectlanguage{ngerman}
\title{2016 Hauptklausur (WS 2015/16)}

\setcounter{section}{1}
%%%%%%%%%%%%%%%%%%%%%%%%%%%%%%%%%%%%%%%%%%%%%%%%%%%%%%%%%%%%%%%%%%%%%%%%%%%%%%
\section*{Aufgabe 1}
\subsection*{Teilaufgabe 1a}
Der Bereich des vom Menschen sichtbaren Lichts liegt zwischen 380 und 700\,nm.
Dort hat Spektrum $B(\lambda)$ die höhere Energie und wird daher heller wahrgenommen.

\subsection*{Teilaufgabe 1b}
\begin{align*}
x &= \frac{X}{X+Y+Z} = 0.5\\
y &= \frac{Y}{X+Y+Z} = 0.3\\
z &= 1 - x - y = 0.2
\end{align*}

\subsection*{Teilaufgabe 1c}
Die Kontrastsensitivität des Menschen ist für Luminanzunterschiede wesentlich stärker ausgeprägt wie für Farbunterschiede.
Die Chrominanzinformation kann daher verlustbehafteter komprimiert werden, ohne dass dies wahrgenommen wird.

\section*{Aufgabe 2}
\subsection*{Teilaufgabe 2a}
\begin{tikzpicture}
	\begin{scope}[shift={(0.7,6.4)},rotate=-25]
		\node[above left] at (0.8,1) {Kamera};
		\fill[gray] (0.8,0) coordinate (cam) ellipse[x radius=0.2,y radius=0.8];
		\draw[thick] (1,-1) -- (0,0) -- (1,1);
	\end{scope}

	\draw (11.5,9) coordinate[label={[yshift=0.7cm]above:Punktlicht}] (light) circle[radius=0.3]
		+(-0.3, 0.3) -- +(-0.6, 0.6) +(0.0, 0.4) -- +(0.0, 0.7) +(0.3, 0.3) -- +(0.6, 0.6)
		+(-0.4, 0.0) -- +(-0.7, 0.0)                            +(0.4, 0.0) -- +(0.7, 0.0)
		+(-0.3,-0.3) -- +(-0.6,-0.6) +(0.0,-0.4) -- +(0.0,-0.7) +(0.3,-0.3) -- +(0.6,-0.6);

	\node[rectangle,draw,fill=gray,anchor=south west,minimum width=14cm, minimum height=6mm] (Sp1) at (0,0) {Spiegel};
	\node[rectangle,draw,fill=gray,anchor=north west,minimum width=9.8cm,minimum height=7mm,rotate=90] (Sp2) at (14,0) {Spiegel};
	\node[rectangle,draw,fill=gray,anchor=north west,minimum width=5.7cm,minimum height=8mm,rotate=90] (D) at (8.3,4.1) {Diffuse Oberfläche};
	\node[rectangle,draw,fill=gray!40,anchor=south west,minimum height=5.5cm,minimum width=2.5cm,align=center,text height=-2cm] (G) at (3.5,1.8) {Glas \\ $\eta > 1$};
	
	\coordinate (TR) at (12,0.6);
	\path[name path=P] (cam) -- +(-30:5cm);
	\path[name path=T] (TR)  -- +(150:10cm);
	\path[name path=Gleft]  (G.north west) -- (G.south west);
	\path[name path=Gright] (G.north east) -- (G.south east);
	\path[name intersections={of=P and Gleft}];	 \coordinate (PT) at (intersection-1);
	\path[name intersections={of=T and Gright}]; \coordinate (TT) at (intersection-1);
	\draw[red] (cam) to[edge label=P] (PT) to[edge label'=T] (TT) to[edge label'=T] (TR) to[edge label=R] (14,1.6) to[edge label'=R] (9.1,4.4) to[edge label=S] (light);
\end{tikzpicture}

\subsection*{Teilaufgabe 2b}
\[\mathbf{d}' = \mathbf{d}\]

\subsection*{Teilaufgabe 2c}
\begin{tabular}{p{10cm}ll}\toprule
	\textbf{Bei einem Whitted-Style Raytracer ist \dots} & \textbf{Notwendig} & \textbf{Optional} \\\midrule
	\dots Primärstrahlen erzeugen                        & \CheckedBox        & \Square           \\
	\dots Strahlschnitte berechnen                       & \CheckedBox        & \Square           \\
	\dots den Fresnel-Term auswerten                     & \Square            & \CheckedBox       \\
	\dots Reflexionsstrahlen rekursiv weiterverfolgen    & \CheckedBox        & \Square           \\
	\dots Mip-Maps erstellen                             & \Square            & \CheckedBox       \\
	\dots Beschleunigungsstrukturen verwenden            & \Square            & \CheckedBox       \\\bottomrule
\end{tabular}

\section*{Aufgabe 3}
\subsection*{Teilaufgabe 3a}
\[M_1 = V_1 \cdot T_A \cdot T_R\]

\subsection*{Teilaufgabe 3b}
\[\mathbf{A}' = T_S^{-1} \cdot T_A \cdot \mathbf{A}\]

\subsection*{Teilaufgabe 3c}
\begin{align*}
V_2 &= T_A^{-1} \cdot T_F^{-1}\\
M_2 &= T_F^{-1} \cdot T_R
\end{align*}

\subsection*{Teilaufgabe 3d}
\begin{tabular}{cp{12cm}ll}\toprule
	~ & \textbf{Aussage} & \textbf{Wahr} & \textbf{Falsch} \\\midrule
	1 & Eine perspektivische Projektion ist immer eine affine Abbildung.
	  & \Square     & \CheckedBox \\
	2 & Zur Transformation von Punkten und dazugehörigen Normalen muss immer die gleiche Transformationsmatrix verwendet werden.
	  & \Square     & \CheckedBox \\
	3 & Homogene Transformationsmatrizen können Translationen darstellen.
	  & \CheckedBox & \Square     \\
	4 & Wenn eine allgemeine Transformation durch eine Matrix $M$ dargestellt wird, so führt $M^\top$ die inverse Transformation durch.
	  & \Square     & \CheckedBox \\
	5 & Bei der homogenen Transformation eines Punktes mit einer Matrix $M$ beschreiben alle $\lambda M$ mit $\lambda \in \mathbb{R}, \lambda \ne 0$ dieselbe Abbildung.
	  & \CheckedBox & \Square     \\\bottomrule
\end{tabular}

\section*{Aufgabe 4}
\subsection*{Teilaufgabe 4a}
\[I = k_a \cdot I_L + k_d \cdot I_L \cdot (\mathbf{n} \cdot \mathbf{l}) + k_s \cdot I_L \cdot (\mathbf{r_l} \cdot \mathbf{v})^n\]

$k_a, k_d, k_s$: Anteil der ambienten/diffusen/spekularen Komponente\\
$I_L$: Intensität der Lichtquelle\\
$n$: Phong-Exponent

\subsection*{Teilaufgabe 4b}
\[\mathbf{r_l} = 2 \cdot \mathbf{n} \cdot (\mathbf{n} \cdot \mathbf{l}) - \mathbf{l}\]

\subsection*{Teilaufgabe 4c}
\begin{description}
	\item[Flat Shading] Berechnung der Flächennormale, Auswertung des Beleuchtungsmodells an einem Punkt für das ganze Dreieck
	\item[Gouraud Shading] Auswertung des Beleuchtungsmodells an den Eckpunkten, dann Interpolation der Ergebnisse
	\item[Phong Shading] Interpolation der Normale, dann Auswertung des Beleuchtungsmodells pro Punkt
	\item[Phong-Beleuchtungsmodell] für alle Shading-Verfahren einsetzbar
\end{description}

\section*{Aufgabe 5}
\subsection*{Teilaufgabe 5a}
\begin{tikzpicture}
	\draw[->] (0,0) -- (13,0) node[below] {x};
	\draw[->] (0,0) --  (0,7) node[left]  {y};
	
	\begin{scope}[every node/.style={rectangle,fill,inner sep=2pt}]
		\draw  (4.8,6.4) node (a) {} +(-0.5,0.2) -- +(0.8, 0.2) -- +(-0.2,-0.5) -- cycle;
		\draw  (7.7,5.2) node (b) {} +(-0.5,0.8) -- +(0.7,-0.2) -- +(-0.2,-0.5) -- cycle;
		\draw (12.1,6.1) node (c) {} +(-0.4,0.3) -- +(0.7, 0.2) -- +(-0.2,-0.5) -- cycle;
		\draw  (2.3,4.2) node (d) {} +(-0.5,0.4) -- +(0.9, 0.1) -- +(-0.4,-0.6) -- cycle;
		\draw (10.4,2.7) node (e) {} +(-0.3,0.9) -- +(0.6,-0.2) -- +(-0.2,-0.7) -- cycle;
		\draw  (3.6,3.7) node (f) {} +(-0.2,0.6) -- +(0.7, 0.0) -- +(-0.6,-0.5) -- cycle;
		\draw  (1.3,1.0) node (g) {} +( 0.0,0.5) -- +(0.7,-0.3) -- +(-0.8,-0.1) -- cycle;
		\draw  (7.1,2.3) node (h) {} +(-0.7,0.5) -- +(0.5, 0.3) -- +( 0.1,-0.8) -- cycle;
	\end{scope}
	
	\begin{scope}[every node/.style={above=2pt,fill=white,inner sep=2pt}]
		\path foreach \p in {a,b,c,d,e,g,h} {(\p) node {\p}};
		\node[below=5pt] at (f) {f};
	\end{scope}

	\begin{scope}[red]
		\draw (0.5,0.7) rectangle (12.8,6.6);
		\draw[dashed] (0.5,0.7) rectangle (5.6,6.6);
		\draw[dotted] (0.5,0.7) rectangle (4.4,4.3);
		\draw[dotted] (1.8,3.6) rectangle (5.6,6.6);
		
		\draw[dashed]  (6.4,1.5) rectangle (12.8,6.4);
		\draw[dotted]  (6.4,1.5) rectangle  (8.4,6.0);
		\draw[dotted] (10.1,2.0) rectangle (12.8,6.4);
	\end{scope}
\end{tikzpicture}

\subsection*{Teilaufgabe 5b}
f, g, a, d

\subsection*{Teilaufgabe 5c}
\begin{tabular}{p{7cm}p{1.5cm}p{1.5cm}p{1.5cm}p{1.5cm}}\toprule
	\textbf{Aussage}                                                                           & \textbf{BVH} & \textbf{kD-Baum} & \textbf{Gitter} & \textbf{keine} \\\midrule
	Die Datenstruktur kann nicht zur Beschleunigung von Nachbarschaftssuchen verwendet werden. & \CheckedBox  & \CheckedBox      & \Square         & \Square        \\
	Der Aufbau der Datenstruktur ist \textit{linear} in der Anzahl der Primitive.              & \Square      & \Square          & \CheckedBox     & \Square        \\
	Mehrfache Schnitttests mit demselben Primitiv müssten explizit vermieden werden.           & \CheckedBox  & \CheckedBox      & \CheckedBox     & \Square        \\
	Die Datenstruktur eignet sich insbesondere für Szenen mit viel freiem Raum.                & \CheckedBox  & \CheckedBox      & \Square         & \Square        \\\bottomrule
\end{tabular}

\subsection*{Teilaufgabe 5d}
\begin{tabular}{p{12cm}ll}\toprule
	\textbf{Die Surface Area Heuristic \dots}                                                & \textbf{Wahr} & \textbf{Falsch} \\\midrule
	\dots wird beim Traversieren einer Datenstruktur eingesetzt.                             & \Square       & \CheckedBox \\
	\dots kann die Traversierung eines kD-Baumes beschleunigen.                              & \CheckedBox   & \Square     \\
	\dots eignet sich besonders für den Aufbau von Oktalbäumen (Octrees).                    & \Square       & \CheckedBox \\
	\dots schätzt die Wahrscheinlichkeit, dass genau $N$ Primitive in einem Teilbaum liegen. & \Square       & \CheckedBox \\\bottomrule
\end{tabular}

\section*{Aufgabe 6}
\subsection*{Teilaufgabe 6a}
\begin{tabular}{cp{12cm}ll}\toprule
	~ & \textbf{Aussage} & \textbf{Wahr} & \textbf{Falsch} \\\midrule
	1 & Texturkoordinaten müssen immer im Bereich $[0, 1]$ liegen.
	  & \Square     & \CheckedBox \\
	2 & Beim Textur-Wrapping kann die Adressierung für jede Dimension separat gewählt werden.
	  & \CheckedBox & \Square     \\
	3 & Anisotrope Texturfilterung sorgt im Allgemeinen dafür, dass Texturen im Vergleich zu isotroper Texturfilterung schärfer erscheinen.
	  & \CheckedBox & \Square     \\
	4 & RIP-Maps können als eine Verallgemeinerung von Mip-Maps angesehen werden.
	  & \CheckedBox & \Square     \\
	5 & Der größte Vorteil des Sphere-Mapping gegenüber Latitude/""Longitude-Maps besteht darin, dass die Abbildung keine Singularität(en) beinhaltet.
	  & \Square     & \CheckedBox \\
	6 & Durch Mip-Mapping wird der Speicheraufwand für Texturen	um den Faktor $\sqrt{2}$ erhöht.
	  & \Square     & \CheckedBox \\
	7 & Bei der Verkleinerung (Minification) werden mehrere	Bildschirm-Pixel auf den selben Texel abgebildet.
	  & \Square     & \CheckedBox \\
	8 & Mittels Summed Area Tables kann man über zwei Texturzugriffe die Summe über eine rechteckige Region in einer Textur berechnen.
	  & \CheckedBox & \Square     \\
	9 & Für die bilineare Filterung muss der Abdruck (Footprint) eines Pixels im Texturraum bestimmt werden.
	  & \CheckedBox & \Square     \\\bottomrule
\end{tabular}

\subsection*{Teilaufgabe 6b}
\textbf{Vorteil:} Abbildung auf Oberflächenmodell einfach (keine Verzerrung etc.)\\
\textbf{Nachteil:} hoher Speicherbedarf

\subsection*{Teilaufgabe 6c}
\begin{description}
	\item[Normal-Mapping] Veränderung der Normalen eines Oberflächenpunkts für die Beleuchtungsberechnung
	\item[Displacement-Mapping] Verschiebung eines Oberflächenpunkts
\end{description}

\section*{Aufgabe 7}
\inputminted[linenos, numbersep=5pt, tabsize=4, frame=lines, label=shader7.frag, mathescape]{glsl}{shader7.frag}

\section*{Aufgabe 8}
\inputminted[linenos, numbersep=5pt, tabsize=4, frame=lines, label=shader8.frag]{glsl}{shader8.frag}

\section*{Aufgabe 9}
TODO

\section*{Aufgabe 10}
\subsection*{Teilaufgabe 10a}
\begin{tikzpicture}[scale=0.75,every circle/.style={radius=2pt}]
	\draw[help lines] (0,0) grid (16,12);
	\draw[thick,->] (0,0) -- (17,0) node[below right] {x};
	\draw[thick,->] (0,0) -- (0,13) node[left] {y};
	\draw foreach \x in {1,...,15} {(\x,0.2) -- (\x,-0.2) node[below] {\x}};
	\draw foreach \y in {1,...,11} {(0.2,\y) -- (-0.2,\y) node[left] {\y}};
	
	\fill  (2,5) coordinate[label=below:$\mathbf{b}_0$] (b0) circle;
	\fill  (3,7) coordinate[label=above:$\mathbf{b}_1$] (b1) circle;
	\fill  (5,8) coordinate[label=above:$\mathbf{b}_2$] (b2) circle;
	\fill  (7,7) coordinate[label=below left:$\mathbf{b}_3 {=} \mathbf{c}_0$] (b3) circle;
	\fill[red]  (9,6) coordinate[label=right:$\mathbf{c}_1$] (c1) circle;
	\fill[red] (11,3) coordinate[label=below:$\mathbf{c}_2$] (c2) circle;
	\fill (14,4) coordinate[label=right:$\mathbf{c}_3$] (c3) circle;
	
	\draw[dashed]     (b0) -- (b1) -- (b2) -- (b3);
	\draw[dashed,red] (b3) -- (c1) -- (c2) -- (c3);
	\draw (b0) .. controls (b1) and (b2) .. (b3) .. controls (c1) and (c2) .. (c3);
\end{tikzpicture}

\subsection*{Teilaufgabe 10b}
\begin{align*}
\mathbf{c}_0 &= \mathbf{b}_3\\
\mathbf{c}_1 &= \mathbf{c}_0 + (\mathbf{b}_3 - \mathbf{b}_2)\\
\mathbf{c}_2 &= 2\mathbf{c}_1 + \mathbf{b}_1 - 2 \mathbf{b}_2
\end{align*}

\subsection*{Teilaufgabe 10c}
TODO


\end{document}