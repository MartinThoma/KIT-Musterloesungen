\documentclass[a4paper]{scrartcl}
\usepackage[ngerman]{babel}
\usepackage[utf8]{inputenc}
\usepackage{amssymb,amsmath}
\usepackage{graphicx}
\usepackage[inline]{enumitem}
\setlist{noitemsep}
\usepackage[binary-units=true]{siunitx}
\usepackage{hyperref}
\usepackage{parskip}
\usepackage[nameinlink,noabbrev,ngerman]{cleveref} % has to be after hyperref
\usepackage{nicefrac}
\usepackage{csquotes}
\usepackage{booktabs}  % for \toprule, \midrule and \bottomrule

\usepackage{minted} % needed for the inclusion of source code

\setcounter{secnumdepth}{2}
\setcounter{tocdepth}{2}

\usepackage{microtype}

\begin{document}
\selectlanguage{ngerman}
\title{2015 Hauptklausur (WS 2014/15)}

\setcounter{section}{1}
%%%%%%%%%%%%%%%%%%%%%%%%%%%%%%%%%%%%%%%%%%%%%%%%%%%%%%%%%%%%%%%%%%%%%%%%%%%%%%
\section*{Aufgabe 1: Raytracing}
\subsection*{Teilaufgabe 1a}
TODO
\subsection*{Teilaufgabe 1b}
TODO
\subsection*{Teilaufgabe 1c}
TODO
\subsection*{Teilaufgabe 1d}
\begin{enumerate}
    \item \textbf{Ray generation}: Erzeuge Sichtstrahlen durch jeden Pixel.
    \item \textbf{Ray intersection}: Schnittberechnung; also: Finde Objekt
          welches den Strahl schneidet und am nahesten zur Kamera ist.
    \item \textbf{Shading}: Schattierung / Beleuchtungsberechnung.
\end{enumerate}

\section*{Aufgabe 2: Farben}
\subsection*{Teilaufgabe 2a}
\textit{Wie nennt man die Funktionen, mit denen man Tristimulus-Werte zu einem gegebenen Spektrum berechnen kann?}

Color Matching Funktionen

\subsection*{Teilaufgabe 2b}
\begin{itemize}
    \item \textit{Es gibt eine lineare Abbildung zwischen den Farbräumen XYZ und xyY.}
    \item[$\rightarrow$] Falsch, da bei der Umrechnung durch $X + Y + Z$ dividiert wird.
    \item \textit{Es gibt eine lineare Abbildung zwischen den Farbräumen RGB und XYZ.}
    \item[$\rightarrow$] Korrekt, der Farbraum wurden konstruiert, um eine möglichst einfache Umrechnung zu ermöglichen.
    \item \textit{Die subjektiv empfundene Stärke von Sinneseindrücken ist proportional zur Intensität des physikalischen Reizes.}
    \item[$\rightarrow$] Falsch, es besteht ein logarithmischer Zusammenhang.
\end{itemize}

\subsection*{Teilaufgabe 2c}
\textit{Welche Information beinhalten die x- und y-Komponenten einer Farbdarstellung im CIE-xyY-Farbraum zusammengenommen?}

Sie beschreiben die Farbe (Chromatizität) unabhängig von der Intensität.

\subsection*{Teilaufgabe 2d}
\textit{Ein RGB-Eingabebild wird in den CIE-xyY-Farbraum transformiert. Um Speicher zu
sparen werden die x- und y-Komponenten in geringerer Auflösung gespeichert, während
die Auflösung der Y-Komponente beibehalten wird.}

\textit{Warum ist dieses Vorgehen im Hinblick auf einen menschlichen Betrachter deutlich
besser, als die Auflösung der RGB-Komponenten zu reduzieren?}

Die Kontrastsentitivität für Chrominanz (Farbe) ist vor allem im hochfrequenten Bereich deutlich geringer als für Luminanz, daher kann man den Chrominanz-Anteil (xy-Werte) in geringerer Auflösung speichern (was dem Weglassen hoher Frequenzen entspricht), ohne dass mit den Augen ein großer Unterschied erkennbar ist.

\section*{Aufgabe 3: Homogene Koordinaten}
\textit{Beim Raytracing soll ein in Weltkoordinaten definierter Strahl mit einem Objekt in dessen
lokalen Koordinatensystem geschnitten werden. Die affine Transformation von lokalen in
Weltkoordinaten ist durch die homogene Transformationsmatrix $M \in \mathbb{R}^{4\times 4}$ gegeben.}
\subsection*{Teilaufgabe 3a}
\textit{Mit welcher Matrix kann der Strahl von Weltkoordinaten in lokale Koordinaten trans-
formiert werden?}

Mit der Matrix $M^{-1}$, da diese die inverse Transformation von Punkten aus \textit{lokalen} in \textit{Weltkoordinaten} darstellt.
\subsection*{Teilaufgabe 3b}
\textit{Der Strahl $\mathbf{R}(t) = \mathbf{E} + t \mathbf{d}, (\mathbf{E}, \mathbf{d} \in \mathbb{R}^3 )$ soll mit der Matrix aus der Teilaufgabe a) in das
lokale Koordinatensystem transformiert werden. Geben Sie dazu den transformierten
Punkt $\mathbf{E'}$ und die transformierte Richtung $\textbf{d'}$ an!}

Zur Vereinfachung geht man zu erweiterten Koordinaten $\begin{pmatrix}
\mathbf{E}\\1
\end{pmatrix}$ und $\begin{pmatrix}
\mathbf{d}\\0
\end{pmatrix}$ über (1 für Punkte, 0 für Richtungen, da diese Differenzen zweier Punkte darstellen).

Damit erhält man $\begin{pmatrix}
\mathbf{E'}\\1
\end{pmatrix} = M^{-1} \begin{pmatrix}
\mathbf{E}\\1
\end{pmatrix}$ und $\begin{pmatrix}
\mathbf{d'}\\0
\end{pmatrix} = M^{-1} \begin{pmatrix}
\mathbf{d}\\0
\end{pmatrix}$
\subsection*{Teilaufgabe 3c}
\textit{Gegeben sei nun}
$$M=\begin{pmatrix}0&0 &-1&5&\\
1&0 &0&0\\
0&-1&0 &0\\
0&0 &0 &1\\
\end{pmatrix}$$
\textit{Berechnen Sie für diese Werte die Matrix aus Aufgabe a)!}

Invertiere die Matrix: $$\left(\begin{array}{cccc|cccc}
0&0 &-1&5&1&0&0&0\\
1&0 &0&0&0&1&0&0\\
0&-1&0 &0&0&0&1&0\\
0&0 &0 &1&0&0&0&1\\
\end{array}\right)\longrightarrow\cdots\longrightarrow\left(\begin{array}{cccc|cccc}
1&0 &0&0&0&1&0&0\\
0&1&0 &0&0&0&-1&0\\
0&0 &1&0&-1&0&0&5\\
0&0 &0 &1&0&0&0&1\\
\end{array}\right)$$

Allgemein gilt für eine Matrix der Form $M = \begin{pmatrix}
A&t\\0&1\end{pmatrix}$: $M^{-1} = \begin{pmatrix}
A^{-1}&-A^{-1}t\\0&1
\end{pmatrix}$

\section*{Aufgabe 4: Transformationen}
\subsection*{Teilaufgabe 4a}
TODO
\subsection*{Teilaufgabe 4b}
\textit{Die folgende Abbildung zeigt ein Quadrat in 2D vor (links) und nach einer Scherung
(rechts).}

TODO

\begin{itemize}
\item[i)] \textit{Geben Sie die Transformationsmatrix $M\in \mathbb{R}^{2\times 2}$ an, die die Scherung beschreibt!}

Die Scherung hat zur Folge, dass beim Erhöhen der x-Koordinate um eine Einheit die y-Koordinate um eine Einheit verringert wird. Damit hat die Matrix die Form
$$M=\begin{pmatrix}
1&0\\-1&1
\end{pmatrix}$$
\item[ii)] \textit{Wie berechnet man allgemein die Matrix $N \in \mathbb{R}^{2\times 2}$ an, die zur Transformation
der Normalen verwendet werden muss? Bestimmen Sie diese Matrix!}

$$N = (M^{-1})^\top = \left(\begin{pmatrix}
1&0\\-1&1
\end{pmatrix}^{-1}\right)^{\top} = \begin{pmatrix}
1&0\\1&1
\end{pmatrix}^{\top} = \begin{pmatrix}
1&1\\0&1
\end{pmatrix}$$

\item[iii)] \textit{Berechnen Sie mithilfe von $N$ die transformierte Normale $\mathbf{n'}$ zur in der Abbildung
eingezeichneten Normalen $\mathbf{n}$!}

$$\mathbf{n'} = \frac{N\mathbf{n}}{||N\mathbf{n}||} = \frac{1}{\sqrt{2}}\begin{pmatrix}
1\\1
\end{pmatrix}$$
\end{itemize}

\section*{Aufgabe 5: Beschleunigungsstrukturen und Hüllkörper}
\subsection*{Teilaufgabe 5a}
TODO
\subsection*{Teilaufgabe 5b}
TODO
\subsection*{Teilaufgabe 5c}
TODO

\section*{Aufgabe 6: Texturen}
\subsection*{Teilaufgabe 6a}
TODO
\subsection*{Teilaufgabe 6b}
TODO

\section*{Aufgabe 7: Beleuchtung}
\subsection*{Teilaufgabe 7a}
$\mathbf{P} = \lambda_1 \mathbf{P}_1 + \lambda_2 \mathbf{P}_2 + \lambda_3 \mathbf{P}_3$, wobei $\lambda_1 + \lambda_2 + \lambda_3 = 1$
und $\lambda_1, \lambda_2, \lambda_3 \geq 0$

\subsection*{Teilaufgabe 7b}
\begin{itemize}
    \item berechne $I_i = k_d I_L (\mathbf{n}_i \cdot normalize(\mathbf{L}-\mathbf{P}_i))$ für $i = 1,2,3$
    \item interpoliere $I = \lambda_1 I_1 + \lambda_2 I_2 + \lambda_3 I_3$
    \item Bemerkung: $normalize(\mathbf{x}) = \frac{\mathbf{x}}{||\mathbf{x}||}$
\end{itemize}

\subsection*{Teilaufgabe 7c}
\begin{itemize}
    \item interpoliere $\mathbf{n} = normalize(\lambda_1 \mathbf{n}_1 + \lambda_2 \mathbf{n}_2 + \lambda_3 \mathbf{n}_3)$
    \item berechne $I = k_d I_L (\mathbf{n} \cdot normalize(\mathbf{L}-\mathbf{P}))$
\end{itemize}

\subsection*{Teilaufgabe 7d}
In den Eckpunkten. Allgemein gilt im $i$-ten Eckpunkt: $\lambda_i = 1$ sowie $\lambda_j = 0 \, (\forall j \neq i)$.
Somit sind die Berechnungen in diesen Punkten äquivalent, wie man leicht aus den obrigen Formeln entnehmen kann.

\subsection*{Teilaufgabe 7e}
Im rechten Fall wird mehr Licht zum betrachter reflektiert.

\subsection*{Teilaufgabe 7f}
Fresnel-Effekt

\section*{Aufgabe 8: Partikeleffekte und OpenGL-Blending}
\subsection*{Teilaufgabe 8a}
TODO
\subsection*{Teilaufgabe 8b}
TODO
\subsection*{Teilaufgabe 8c}
TODO
\subsection*{Teilaufgabe 8d}
TODO

\section*{Aufgabe 9: OpenGL}
TODO

\section*{Aufgabe 10: Reflexionen in OpenGL}
\inputminted[linenos, numbersep=5pt, tabsize=4, frame=lines, label=shader.frag]{glsl}{shader.frag}

\section*{Aufgabe 11: GLSL-Hatching}
\subsection*{Teilaufgabe 11a}

\[
	\left(1 - \frac{x - x_1}{x_2 - x_1}\right) y_1 + \frac{x - x_1}{x_2 - x_1} y_2
\]

oder

\[
	y_1 + \frac{y_2 - y_1}{x_2 - x_1} (x - x_1)
\]

\subsection*{Teilaufgabe 11b}
\inputminted[linenos, numbersep=5pt, tabsize=4, frame=lines, label=shader.frag]{glsl}{hatching.frag}

\section*{Aufgabe 12: Bézierkurven}
\subsection*{Teilaufgabe 12a}
TODO
\subsection*{Teilaufgabe 12b}
TODO
\subsection*{Teilaufgabe 12c}
\begin{enumerate}
    \item Ja
    \item Ja
    \item Nein, da die Kurve nicht innerhalb der konvexen Hülle der
          Kontrollpunkte ist.
\end{enumerate}

\end{document}
