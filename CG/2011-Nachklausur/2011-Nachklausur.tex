\documentclass[a4paper]{scrartcl}
\usepackage[ngerman]{babel}
\usepackage[utf8]{inputenc}
\usepackage{amssymb,amsmath}
\usepackage{graphicx}
\usepackage[inline]{enumitem}
\setlist{noitemsep}
\usepackage[binary-units=true]{siunitx}
\usepackage{hyperref}
\usepackage{parskip}
\usepackage[nameinlink,noabbrev,ngerman]{cleveref} % has to be after hyperref
\usepackage[colorinlistoftodos]{todonotes}
\usepackage{nicefrac}  % for \nicefrac{1}{3}
\usepackage{csquotes}  % for \enquote{what you want to quote}
\usepackage{booktabs}  % for \toprule, \midrule and \bottomrule
\usepackage{minted} % needed for the inclusion of source code

% for \begin{enumerate}[label=(\Alph*)], see http://tex.stackexchange.com/a/129960/5645
\usepackage{enumitem}

\setcounter{secnumdepth}{2}
\setcounter{tocdepth}{2}

\usepackage{microtype}

% \begin{figure}[h]
%     \centering
%     \includegraphics*[width=0.8\linewidth, keepaspectratio]{1a.png}
%     \caption{Whatever}
%     \label{fig:1a}
% \end{figure}

\begin{document}
\selectlanguage{ngerman}
\title{2011 Nachklausur (WS 2010/11)}

\setcounter{section}{1}
%%%%%%%%%%%%%%%%%%%%%%%%%%%%%%%%%%%%%%%%%%%%%%%%%%%%%%%%%%%%%%%%%%%%%%%%%%%%%%
\section*{Aufgabe 1: Wahrnehmung, Farbe und Rasterbilder}
\subsection*{Teilaufgabe 1a}
\textit{Was versteht man unter Metamerie beim Farbsehen des Menschen?}

Metamerie ist das Phänomen, dass verschiedene Spektren den selben Farbeindruck
erzeugen können.

\subsection*{Teilaufgabe 1b}
\textit{Was versteht man unter Schwarzkörperstrahlung und Farbtemperatur?}

Ein Schwarzkörper ist eine idealisierte thermische Strahlungsquelle. Die
idealisierung besteht darin, dass der Körper die komplette auftretende
Strahlung vollständig absorbiert. Gleichzeitig sendet er Wärmestrahlung
(Schwarzkörperstrahlung) aus, welche nur von seiner Temperatur abhängig ist.

Die Farbtemperatur ist ein Maß, um einen jeweiligen Farbeindruck einer
Lichtquelle zu bestimmen.

\subsection*{Teilaufgabe 1c}
Siehe \href{https://martin-thoma.com/html5/graphic-filters/graphic-filters.htm}{martin-thoma.com/html5/graphic-filters} zum ausprobieren.

\begin{enumerate}[label=(\Alph*)]
    \item Hervorheben von horizontalen Kanten.
    \item Unschärfe / Weichzeichnen
    \item Hervorheben aller Kanten
    \item Hervorheben aller Kanten (Invertierter Laplace-Filter), entfernen vom
          Rest
\end{enumerate}

\subsection*{Teilaufgabe 1d}
\textit{Was versteht man unter einem normalisierten Filterkernel?}

Ein normalisierter Filterkernel hat als Summe der Element den Wert~1.

\textit{Welche globale Eigenschaft eines Bildes ändert sich, wenn ein Filterkernel nicht normalisiert ist?}

Die Helligkeit des Bildes ändert sich nicht.

\clearpage
\section*{Aufgabe 2: Prozedurale Modelle}
\subsection*{Teilaufgabe 2a}
\textit{Was sind Turbulenzfunktionen und wie können Sie aus Noise-Funktionen gebildet werden?}

Eine Turbulenzfunktion summiert $k$ Oktaven mehrerer Noise-Funktionen $n$ auf:

\[\text{turbulence}(x) = \sum_k \left (\frac{1}{2} \right )^k \cdot n(2^k \cdot x)\]

Einsatzgebiete:

\begin{itemize}
    \item Natürliche Oberflächen
    \item Feuer
\end{itemize}

\subsection*{Teilaufgabe 2b: Kontext-Freie Lindenmayer-Systeme}

\begin{enumerate}[label=(\arabic*)]
    \item $F \;\;\;\rightarrow\;\;\;F [+F][-F]\;\;\;\rightarrow\;\;\;F [+F [+F][-F]] [-F [+F][-F]]$
    \item $F \;\;\;\rightarrow\;\;\;F[+F]\;\;\;\rightarrow\;\;\;F[+F][+F[+F]]$
    \item $F \;\;\;\rightarrow\;\;\;F [f - F] fF\;\;\;\rightarrow\;\;\;F [f - F] fF [f - F [f - F] fF] fF [f - F] fF$
\end{enumerate}

\subsection*{Teilaufgabe 2c: Turtle-Grafiken}

\begin{itemize}
    \item Die Grafik links oben ist (1)
    \item Die Grafik in der Mitte, unten ist (2)
    \item Die Grafik rechts unten ist (3)
\end{itemize}

\section*{Aufgabe 3: Supersampling und Baryzentrische Koordinaten}
\subsection*{Teilaufgabe 3a}
\textit{Was ist adaptives Supersampling?}

Beim adaptiven Supersampling wird durch zwei benachbarte Pixel jeweils ein
Strahl geschossen. Ist die Differenz der Pixelwerte über einem Schwellwert, so
schießt man weitere Strahlen zwischen den beiden Pixeln. Dies wiederholt man
so lange, bis man unter dem Schwellwert ist.

\textit{Was ist stochastisches Supersampling mit Stratifikation?}

Beim stochastischen Supersampling wird jeder Pixel in ein Gitter unterteilt
und durch jeden Gitterpunkt wird mit einer gewissen Wahrscheinlichkeit ein
Strahl geschossen.

\textit{Was sind die Unterschiede zwischen adaptivem Supersampling und stochastischem
Supersampling mit Stratifikation?}

Adaptives Supersampling schießt nur bei Bedarf weitere Strahlen. Allerdings
kann man Fälle konstruieren, wo adaptives Supersampling immer fehlschlägt.

\subsection*{Teilaufgabe 3b}

\begin{align}
    \lambda_A &= \frac{A_\Delta(P,B,C)}{A_\Delta(A,B,C)} = \frac{2}{7.5} = \frac{4}{15}\\
    \lambda_B &= \frac{A_\Delta(P,A,C)}{A_\Delta(A,B,C)} = \frac{2.5}{7.5} = \frac{5}{15}\\
    \lambda_C &= \frac{A_\Delta(P,A,B)}{A_\Delta(A,B,C)} = \frac{3}{7.5} = \frac{6}{15}
\end{align}

\section*{Aufgabe 4: Texturen}
\subsection*{Teilaufgabe 4a}
\textit{Wie wird aus einer Textur eine Mip-Map-Pyramide erstellt?}

TODO

\textit{Wie hoch ist der zusätzliche Speicherbedarf?}

Der zustätzliche Speicherbedarf $z$ ist $\approx \nicefrac{1}{3}$ der ursprünglichen
Texturgröße.

Beweis:

Es gilt:
\[z = \sum_{i=1}^\infty \left (\frac{1}{4} \right )^i\]

Dies ist eine geometrische Reihe. Die Summe der ersten $n$ Terme ist

\[\frac{1- (\nicefrac{1}{4})^n}{1 - \nicefrac{1}{4}}\]

daher gilt:

\[\lim_{n \rightarrow \infty} = 1 - \frac{1}{1 - \nicefrac{1}{4}} = \nicefrac{1}{3}\]

\subsection*{Teilaufgabe 4b}
\textit{Welche Probleme bei der Texturfilterung im Fall der Verkleinerung (Texture Minification) löst Mip-Mapping?}

Mip-Mapping verringert Aliasing-Effekte.

\textit{Welche Probleme bei der Texturfilterung im Fall der Verkleinerung löst Mip-Mapping nicht?}

Verwaschenes aussehen bei länglichem Footprint, da Mip-Map isotrop (TODO: erklären, ausformulieren).

\subsection*{Teilaufgabe 4c}
\textit{Wofür verwendet man Environment Mapping?}

TODO

\textit{Was speichert eine Environment Map?}

TODO

\textit{Welche vereinfachenden Annahmen werden bei der Anwendung getroffen?}

Der Betrachter ist sehr weit von der Umgebung entfernt, sodass die Position
keine Rolle spielt und ausschließlich die Blickrichtung wichtig ist.


\section*{Aufgabe 5}
TODO

\section*{Aufgabe 6}
TODO

\section*{Aufgabe 7}
TODO

\section*{Aufgabe 8}
TODO

\section*{Aufgabe 9}
TODO

% \inputminted[linenos, numbersep=5pt, tabsize=4, frame=lines, label=shader.frag]{glsl}{shader.frag}


\end{document}
