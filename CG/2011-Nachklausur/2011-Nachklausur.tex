\documentclass[a4paper]{scrartcl}
\usepackage[ngerman]{babel}
\usepackage[utf8]{inputenc}
\usepackage{amssymb,amsmath}
\usepackage{graphicx}
\usepackage[inline]{enumitem}
\setlist{noitemsep}
\usepackage[binary-units=true]{siunitx}
\usepackage{hyperref}
\usepackage{parskip}
\usepackage[nameinlink,noabbrev,ngerman]{cleveref} % has to be after hyperref
\usepackage{nicefrac}  % for \nicefrac{1}{3}
\usepackage{csquotes}  % for \enquote{what you want to quote}
\usepackage{booktabs}  % for \toprule, \midrule and \bottomrule
\usepackage{minted} % needed for the inclusion of source code

% for \begin{enumerate}[label=(\Alph*)], see http://tex.stackexchange.com/a/129960/5645
\usepackage{enumitem}

\setcounter{secnumdepth}{2}
\setcounter{tocdepth}{2}

\usepackage{wasysym}  % For \CheckedBox
\usepackage{microtype}

% \begin{figure}[h]
%     \centering
%     \includegraphics*[width=0.8\linewidth, keepaspectratio]{1a.png}
%     \caption{Whatever}
%     \label{fig:1a}
% \end{figure}

\begin{document}
\selectlanguage{ngerman}
\title{2011 Nachklausur (WS 2010/11)}

\setcounter{section}{1}
%%%%%%%%%%%%%%%%%%%%%%%%%%%%%%%%%%%%%%%%%%%%%%%%%%%%%%%%%%%%%%%%%%%%%%%%%%%%%%
\section*{Aufgabe 1: Wahrnehmung, Farbe und Rasterbilder}
\subsection*{Teilaufgabe 1a}
\textit{Was versteht man unter Metamerie beim Farbsehen des Menschen?}

Metamerie ist das Phänomen, dass verschiedene Spektren den selben Farbeindruck
erzeugen können.

\subsection*{Teilaufgabe 1b}
\textit{Was versteht man unter Schwarzkörperstrahlung und Farbtemperatur?}

Ein Schwarzkörper ist eine idealisierte thermische Strahlungsquelle. Die
idealisierung besteht darin, dass der Körper die komplette auftretende
Strahlung vollständig absorbiert. Gleichzeitig sendet er Wärmestrahlung
(Schwarzkörperstrahlung) aus, welche nur von seiner Temperatur abhängig ist.

Die Farbtemperatur ist ein Maß, um einen jeweiligen Farbeindruck einer
Lichtquelle zu bestimmen.

\subsection*{Teilaufgabe 1c}
Siehe \href{https://martin-thoma.com/html5/graphic-filters/graphic-filters.htm}{martin-thoma.com/html5/graphic-filters} zum ausprobieren.

\begin{enumerate}[label=(\Alph*)]
    \item Hervorheben von horizontalen Kanten.
    \item Unschärfe / Weichzeichnen
    \item Hervorheben aller Kanten (Schärfen; vgl. Übung 02\_Bildoperationen, Folie 19)\footnote{Siehe \href{https://martin-thoma.com/html5/graphic-filters/graphic-filters.htm?k1=0&k2=-1&k3=0&k4=-1&k5=5&k6=-1&k7=0&k8=-1&k9=0}{martin-thoma.com/html5/graphic-filters/graphic-filters.htm}}
    \item Hervorheben aller Kanten (Invertierter Laplace-Filter), entfernen vom
          Rest
\end{enumerate}

\subsection*{Teilaufgabe 1d}
\textit{Was versteht man unter einem normalisierten Filterkernel?}

Ein normalisierter Filterkernel hat als Summe der Element den Wert~1.

\textit{Welche globale Eigenschaft eines Bildes ändert sich, wenn ein Filterkernel nicht normalisiert ist?}

Die Gesamthelligkeit des Bildes ändert sich nicht (vgl. Übungsfolie \texttt{02\_Bildoperationen} Folie 19)

\clearpage
\section*{Aufgabe 2: Prozedurale Modelle}
\subsection*{Teilaufgabe 2a}
\textit{Was sind Turbulenzfunktionen und wie können Sie aus Noise-Funktionen gebildet werden?}

Eine Turbulenzfunktion summiert $k$ Oktaven mehrerer Noise-Funktionen $n$ auf:

\[\text{turbulence}(x) = \sum_k \left (\frac{1}{2} \right )^k \cdot n(2^k \cdot x)\]

Einsatzgebiete:

\begin{itemize}
    \item Natürliche Oberflächen
    \item Feuer
\end{itemize}

\subsection*{Teilaufgabe 2b: Kontext-Freie Lindenmayer-Systeme}

\begin{enumerate}[label=(\arabic*)]
    \item $F \;\;\;\rightarrow\;\;\;F [+F][-F]\;\;\;\rightarrow\;\;\;F [+F [+F][-F]] [-F [+F][-F]]$
    \item $F \;\;\;\rightarrow\;\;\;F[+F]\;\;\;\rightarrow\;\;\;F[+F][+F[+F]]$
    \item $F \;\;\;\rightarrow\;\;\;F [f - F] fF\;\;\;\rightarrow\;\;\;F [f - F] fF [f - F [f - F] fF] fF [f - F] fF$
\end{enumerate}

\subsection*{Teilaufgabe 2c: Turtle-Grafiken}

\begin{itemize}
    \item Die Grafik links oben ist (1)
    \item Die Grafik in der Mitte, unten ist (2)
    \item Die Grafik rechts unten ist (3)
\end{itemize}

\section*{Aufgabe 3: Supersampling und Baryzentrische Koordinaten}
\subsection*{Teilaufgabe 3a}
\textit{Was ist adaptives Supersampling?}

Beim adaptiven Supersampling wird durch zwei benachbarte Pixel jeweils ein
Strahl geschossen. Ist die Differenz der Pixelwerte über einem Schwellwert, so
schießt man weitere Strahlen zwischen den beiden Pixeln. Dies wiederholt man
so lange, bis man unter dem Schwellwert ist.

\textit{Was ist stochastisches Supersampling mit Stratifikation?}

Beim stochastischen Supersampling wird jeder Pixel in ein Gitter unterteilt
und durch jeden Gitterpunkt wird mit einer gewissen Wahrscheinlichkeit ein
Strahl geschossen.

\textit{Was sind die Unterschiede zwischen adaptivem Supersampling und stochastischem
Supersampling mit Stratifikation?}

Adaptives Supersampling schießt nur bei Bedarf weitere Strahlen. Allerdings
kann man Fälle konstruieren, wo adaptives Supersampling immer fehlschlägt.

\subsection*{Teilaufgabe 3b}

\begin{align}
    \lambda_A &= \frac{A_\Delta(P,B,C)}{A_\Delta(A,B,C)} = \frac{2}{7.5} = \frac{4}{15}\\
    \lambda_B &= \frac{A_\Delta(P,A,C)}{A_\Delta(A,B,C)} = \frac{2.5}{7.5} = \frac{5}{15}\\
    \lambda_C &= \frac{A_\Delta(P,A,B)}{A_\Delta(A,B,C)} = \frac{3}{7.5} = \frac{6}{15}
\end{align}

\section*{Aufgabe 4: Texturen}
\subsection*{Teilaufgabe 4a}
\textit{Wie wird aus einer Textur eine Mip-Map-Pyramide erstellt?}

Man erzeugt Mip-Maps durch \enquote{zusammenfassen} von jeweilse $2 \times 2$
Texeln. Es werden solange neue Mip-Map-Stufen generiert, bis man eine Mip-Map
erzeugt hat, die nur noch aus einem Texel besteht. In Stufe 0 wird die
ursprüngliche Textur gespeichert, in Stufe 1 dann eine Textur die in beiden
Dimensionen halbiert wurde usw.

\textit{Wie hoch ist der zusätzliche Speicherbedarf?}

Der zustätzliche Speicherbedarf $z$ ist $\approx \nicefrac{1}{3}$ der ursprünglichen
Texturgröße.

Beweis:

Es gilt:
\[z = \sum_{i=1}^\infty \left (\frac{1}{4} \right )^i\]

Dies ist eine geometrische Reihe. Die Summe der ersten $n$ Terme ist
\[\frac{1- (\nicefrac{1}{4})^n}{1 - \nicefrac{1}{4}}\]

daher gilt:
\[\lim_{n \rightarrow \infty} = 1 - \frac{1}{1 - \nicefrac{1}{4}} = \nicefrac{1}{3}\]

\subsection*{Teilaufgabe 4b}
\textit{Welche Probleme bei der Texturfilterung im Fall der Verkleinerung (Texture Minification) löst Mip-Mapping?}

Mip-Mapping verringert Aliasing-Effekte.

\textit{Welche Probleme bei der Texturfilterung im Fall der Verkleinerung löst Mip-Mapping nicht?}

Verwaschenes aussehen bei länglichem Footprint, da Mip-Map isotrop (TODO: erklären, ausformulieren).

\subsection*{Teilaufgabe 4c}
\textit{Wofür verwendet man Environment Mapping?}

Darstellung reflektierender Objekte mit Spiegelung von Umgebung ohne
geometrische Repräsentation.

\textit{Was speichert eine Environment Map?}

Beleuchtungsinformationen

\textit{Welche vereinfachenden Annahmen werden bei der Anwendung getroffen?}

Der Betrachter ist sehr weit von der Umgebung entfernt, sodass die Position
keine Rolle spielt und ausschließlich die Blickrichtung wichtig ist.


\section*{Aufgabe 5: $\alpha$-Clipping}
\subsection*{Teilaufgabe 5a}

\begin{align}
    \text{WEC}_{x_\text{min}}(P_0) &= -3-0 =-3 & \text{WEC}_{x_\text{min}}(P_1) &= 6-0 =6\\
    \text{WEC}_{x_\text{max}}(P_0) &= 10+3 =13 & \text{WEC}_{x_\text{max}}(P_1) &= 10-6 =4\\
    \text{WEC}_{y_\text{min}}(P_0) &= 4-0 = 4 & \text{WEC}_{y_\text{min}}(P_1) &= -2-0 =-2\\
    \text{WEC}_{y_\text{max}}(P_0) &= 6-4 = 2 & \text{WEC}_{y_\text{max}}(P_1) &= 6+2 =8
\end{align}

\goodbreak
\subsection*{Teilaufgabe 5b}
\textit{Wenn Sie alleine die WEC betrachten, welche Kanten des Viewports werden dann potenziell geschnitten? Begründen Sie Ihre Antwort mit Hilfe der Outcodes!}\\
Da für $\text{WEC}_{x_\text{min}}(P_0) < 0$ und $\text{WEC}_{y_\text{min}}(P_1) < 0$
ist der Outcode $x_\text{min}(P_0) = 1$ und Outcode $y_\text{min}(P_1) = 1$.
Daraus folgt, dass $x_\text{min}$ und $y_\text{min}$ auf Schnitt getestet
werden müssen. (TODO: Stimmt das?)

\subsection*{Teilaufgabe 5c}
\begin{align}
    a_{max} &= 1,\;\;\; a_{min} = 0\\
    a_{min} &= \max(a_{min}, a_s) = \nicefrac{1}{3}\\
    a_s &= \frac{\text{WEC}_{y_\text{min}}(P_1)}{\text{WEC}_{y_\text{min}}(P_0) - \text{WEC}_{y_\text{min}}(P_1)} = \nicefrac{4}{6} = \nicefrac{2}{3}\\
    a_{max} &= \min(a_{max}, a_s) = \nicefrac{2}{3}
\end{align}

Das neue Liniensegment ist somit $(P_0 + \nicefrac{1}{3} \cdot (P_1- P_0), P_1 - \nicefrac{2}{3} \cdot (P_1- P_0))$.

\section*{Aufgabe 6}
\subsection*{Teilaufgabe 6a}
\textit{Was ist der Unterschied zwischen Gouraud- und Phong-Shading bei der Schattierung eines Dreiecks mit einem Rasterisierungsverfahren?}

TODO

\subsection*{Teilaufgabe 6b}
\textit{Was versteht man unter einem Accumulation-Buffer, wie Sie ihn von OpenGL kennen?}

Der Accumulation-Buffer ist ein Zwischenspeicher zur Kombination mehrerer
Redineringschritte.

\textit{Nennen Sie zwei Beispiele für Effekte, die sich mit einem Accumulation-Buffer erreichen lassen!}

\begin{itemize}
    \item Bewegungsunschärfe
    \item TODO
\end{itemize}

\subsection*{Teilaufgabe 6c}

\begin{tabular}{p{8cm}ccp{4cm}}\toprule
Aussage & Wahr & Falsch & Begründung \\\midrule
T-Vertices können bei Phong-Shading Artefakte verursachen. & \CheckedBox & $\square$ & TODO: Wirklich? \\
Z-Fighting kann durch die Repräsentation der Tiefenwerte mit beschränkter Genauigkeit im Tiefenpuffer entstehen. & \CheckedBox & $\square$ & TODO: Wirklich?          \\
Je feiner eine Oberfläche tesselliert wird, umso geringer werden die Unterschiede zwischen Gouraud- und Phong-Shading. & \CheckedBox & $\square$ & TODO: Wirklich?          \\
Bei der Rasterisierung ist eine perspektivisch korrekte Abbildung der Textur aufwendiger als eine affine, da pro Pixel eine zusätzliche Division benötigt wird. & $\square$ & $\square$ & TODO          \\
Der Scissor-Test dient dazu, durchsichtige Teile einer Oberfläche gemäß einer Textur wegzuschneiden. & $\square$ & \CheckedBox & Dient dazu Teile auserhalb des Rechtecks wegzuschneiden (vgl. OpenGL Teil 2, Folie 103) \\
Screendoor-Transparency stellt transparente Objekte mittels Blending dar. & $\square$ & $\square$ & TODO \\\bottomrule
\end{tabular}


\section*{Aufgabe 7: OpenGL}
\subsection*{Teilaufgabe 7a: Blending}
\subsection*{Teilaufgabe 7a (1)}

Es erfolgt kein Blending. Durch den aktivierten Tiefentest wird der rote Würfel
nicht beachtet, da er hinter dem blauen Ball liegt. Daher:

\texttt{color(P) = (0.0, 0.0, 1.0, 0.5)}

siehe \enquote{OpenGL Teil 2 und 3}, Folie 88.

\subsection*{Teilaufgabe 7a (2)}
\begin{align}
    c' &= 1 \cdot (0,0,1,0.5) + 0.5 \cdot (0,0,0,0) = (0,0,1,0.5)\\
    c  &= 1 \cdot (1,0,0,0.5) + 0.5 \cdot (0,0,1,0.5) = (1,0,0.5,0.75)
\end{align}

\subsection*{Teilaufgabe 7b}
\textit{Bringen Sie die folgende Operationen in die richtige Reihenfolge, wie sie in der Fixed-
Function-Pipeline von OpenGL ausgeführt werden:}

Die Fixed Function Pipeline kann nur Gouraud oder Flat Shading. Das heißt, die
Beleuctung wird pro Vertex in Kamera Koordinaten berechnet. Dementsprechend:
\begin{enumerate}
    \item Model-View-Transformation anwenden (1)
    \item Projektionstransformation anwenden (4)
    \item Beleuchtungsberechnung (5)
    \item Texturierung (3)
    \item Tiefentest (2)
\end{enumerate}

Leider sind wir uns bei dieser Lösung ziemlich unsicher (siehe \href{https://github.com/MartinThoma/KIT-Musterloesungen/issues/22}{GitHub issue(s)}).

\begin{itemize}
    \item Beleuchtungsberechnung in Kamerakoordinaten nach der
          Modelview-Transformation (Folie 39 $\Rightarrow$ Erst (1), dann (5))
\end{itemize}

\subsection*{Teilaufgabe 7c}
Zwischen 2 und 3, da die Rasterisierung nur für Dreiecke erfolgen soll, die
im Viewfrustum liegen.

\section*{Aufgabe 8}
\subsection*{Teilaufgabe 8a}
\begin{tabular}{p{8cm}p{1.5cm}p{1.5cm}p{1.5cm}}\toprule
Operation                          & Vertex-Shader & Fragment-Shader & Weder noch \\\midrule
Model-View-Transformation anwenden & $\square$     & $\square$       & $\square$ TODO\\ % \CheckedBox
Tiefentest                         & $\square$     & $\square$       & $\square$ TODO\\
Texturierung                       & $\square$     & $\square$       & $\square$ TODO\\
Projektionstransformation anwenden & $\square$     & $\square$       & $\square$ TODO\\
Beleuchtungsberechnung             & $\square$     & $\square$       & $\square$ TODO\\
Clipping                           & $\square$     & $\square$       & $\square$ TODO\\\bottomrule
\end{tabular}

\subsection*{Teilaufgabe 8b}
\begin{itemize}
    \item \texttt{attribute}: Attribut eines Vertex; nur für Vertex-Shader; z.B. Farbe oder Normale
    \item \texttt{uniform}: Bei jedem Shader-Aufruf gleich (also insbesondere für jeden Vertex gleich); read-only; z.B. Transformationsmatrix
    \item \texttt{varying}: weitergegebene/interpolierte Werte (schreiben in einem Shader, lesen im darauffolgenden Shader)
\end{itemize}

\subsection*{Teilaufgabe 8c}
\inputminted[linenos, numbersep=5pt, tabsize=4, frame=lines, label=shader.vert]{glsl}{shader.vert}

\inputminted[linenos, numbersep=5pt, tabsize=4, frame=lines, label=shader.frag]{glsl}{shader.frag}


\section*{Aufgabe 9: Bézier-Kurven}
\subsection*{Teilaufgabe 9a}

\begin{align}
     & b_0 \cdot (1-2u + u^2) + b_1 (2u - 2u^2) + b_2 \cdot u^2\\
    =& b_0 - 2 b_0 u + b_0 u^2 + 2 b_1 u - 2 b_1 u^2 + b_2 u^2\\
    &= \underbrace{(b_0 - 2 b_1 - b_2)}_{= a_2} u^2  + \underbrace{(2 b_1 - 2 b_0)}_{= a_1} u + \underbrace{b_0}_{= a_0}
\end{align}

\subsection*{Teilaufgabe 9b}
\textit{Was versteht man unter affiner Invarianz der Bézier-Repräsentation?}

Für jede Bézierkurve $F(u)$ und jede affine Abbildung $\varphi(x) = A x + t$ gilt:

\[\varphi(F(u)) = \sum_{i=0}^n B_i^n(u) \varphi(\mathbf{b}_i)\]

Das heißt es genügt bei Transformationen die Kontrollpunkte $\mathbf{b}_i$
zu transformieren.

\subsection*{Teilaufgabe 9c}
Für $C^0$-Stetigkeit muss $c_0 = b_3 = (4, 1)$ gelten.

Für $C_1$-Stetigkeit muss $b_2 - b_3 = c_0 - c_1 \Leftrightarrow (3,1) - (4,1) = (4,1) = \underbrace{(5,1)}_{= c_1}$ gelten.

Für $C_2$-Stetigkeit muss

\begin{align}
    b_2 + (b_2 - b_1) &= c_1 + (c_1 - c_2)\\
    \Leftrightarrow (3,1) + ((3,1) - (1,2)) &= (5,1) + ((5,1) - c_2))\\
    \Leftrightarrow (5, 0) &= (10, 2) - c_2\\
    \Leftrightarrow c_2 &= (5, 2)
\end{align}
\end{document}
