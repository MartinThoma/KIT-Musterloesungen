\documentclass[a4paper]{scrartcl}
\usepackage[ngerman]{babel}
\usepackage[utf8]{inputenc}
\usepackage{amssymb,amsmath}
\usepackage{graphicx}
\usepackage[inline]{enumitem}
\setlist{noitemsep}
\usepackage[binary-units=true]{siunitx}
\usepackage{hyperref}
\usepackage{parskip}
\usepackage[nameinlink,noabbrev,ngerman]{cleveref} % has to be after hyperref
\usepackage[colorinlistoftodos]{todonotes}
\usepackage{nicefrac}
\usepackage{csquotes}

\usepackage{minted} % needed for the inclusion of source code

\setcounter{secnumdepth}{2}
\setcounter{tocdepth}{2}

\usepackage{microtype}

\begin{document}
\selectlanguage{ngerman}
\title{2014 Nachklausur (WS 2013/14)}

\setcounter{section}{1}
%%%%%%%%%%%%%%%%%%%%%%%%%%%%%%%%%%%%%%%%%%%%%%%%%%%%%%%%%%%%%%%%%%%%%%%%%%%%%%
\section*{Aufgabe 1: Das Phong-Beleuchtungsmodell}
\subsection*{Teilaufgabe 1a}
\begin{figure}[h]
    \centering
    \includegraphics*[width=0.8\linewidth, keepaspectratio]{1a.png}
    \caption{Skizze zu Aufgabe 1}
    \label{fig:1a}
\end{figure}

\subsection*{Teilaufgabe 1b}
TODO
\subsection*{Teilaufgabe 1c}
TODO
\subsection*{Teilaufgabe 1d}
TODO
\subsection*{Teilaufgabe 1e}
TODO
\subsection*{Teilaufgabe 1f}
TODO

\section*{Aufgabe 2: Raytracing}
\subsection*{Teilaufgabe 2a}
\begin{itemize}
    \item Anstelle einen Punkt für einen Pixel abzutasten, tastet man
          $k^2$ mal in äquidistanten Intervallen ab.
    \item Aliasing wird dadurch verringert.
\end{itemize}

\subsection*{Teilaufgabe 2b}
\begin{itemize}
    \item Maximale Rekursionstiefe erreicht
    \item Rekursion bis der Beitrag zur Farbe vernachlässigbar wird
\end{itemize}

\subsection*{Teilaufgabe 2c}

\textit{Was ist der Unterschied zwischen Distributed Raytracing und Whitted-Style Raytracing?}
TODO

\textit{Welchen Lichttransport kann man durch Distributed Raytracing berechnen, den
Whitted-Style Raytracing nicht erfassen kann?}
TODO

\subsection*{Teilaufgabe 2d}
\textit{Nennen Sie kurz und stichpunktartig die zwei Schritte, die zur Berechnung von Vertex-
Normalen bei einem Dreiecksnetz notwendig sind! Gehen Sie dabei davon aus, dass nur
die Vertex-Positionen und die Topologie des Netzes gegeben sind!}
TODO

\section*{Aufgabe 3}
\subsection*{Teilaufgabe 3a}
TODO
\subsection*{Teilaufgabe 3b}
TODO

\section*{Aufgabe 4}
\subsection*{Teilaufgabe 4a}
TODO
\subsection*{Teilaufgabe 4b}
TODO

\section*{Aufgabe 5}
TODO

\section*{Aufgabe 6: Texturierung}
\subsection*{Teilaufgabe 6a}
TODO
\subsection*{Teilaufgabe 6b}
TODO
\subsection*{Teilaufgabe 6c}
TODO
\subsection*{Teilaufgabe 6d}
TODO

\section*{Aufgabe 7: Cube-Maps und Environment-Mapping}
\subsection*{Teilaufgabe 7a}
TODO
\subsection*{Teilaufgabe 7b}
TODO

\section*{Aufgabe 8: Hierarchische Datenstrukturen}
\subsection*{Teilaufgabe 8a}
TODO
\subsection*{Teilaufgabe 8b}
TODO
\subsection*{Teilaufgabe 8c}
TODO
\subsection*{Teilaufgabe 8d}
TODO

\section*{Aufgabe 9: Rasterisierung und OpenGL}
TODO

\section*{Aufgabe 10: Tiefenpuffer und Transparenz}
\subsection*{Teilaufgabe 10a}
TODO
\subsection*{Teilaufgabe 10b}
TODO
\subsection*{Teilaufgabe 10c}
TODO

\section*{Aufgabe 11: Phong-Shading und Phong-Beleuchtungsmodell}
\inputminted[linenos, numbersep=5pt, tabsize=4, frame=lines, label=shader.vert]{glsl}{shader.vert}


\end{document}
